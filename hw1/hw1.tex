\documentclass[10pt]{article}

\usepackage{amsfonts, amsthm, fullpage}

\newcommand{\card}[1]{\left| #1 \right|}
\newcommand{\nat}{\mathbb{N}}
\newcommand{\ints}{\mathbb{Z}}
\newcommand{\reals}{\mathbb{R}}

\begin{document}
\begin{flushleft}
\textbf{Geoffrey Parker - grp352 \\
CS 341 -53005 \\
Homework 1}
\end{flushleft}
This assignment covers the review material in Appendix A.
\begin{enumerate}

%--
% 1
%--

\item
Prove each of the following:
\begin{enumerate}
\item
$((A \land B) \rightarrow C) \leftrightarrow (\lnot A \lor \lnot B \lor C)$. \\
\begin{proof}
Assume $(A \land B) \rightarrow C$. \\ \\
\begin{tabular}{r c l}
1) & $(A \land B) \rightarrow C$ & Assumption \\
2) & $\lnot (A \land B) \lor C$ & 1 and Implication \\
3) & $(\lnot A \lor \lnot B) \lor C$ & 2 and DeMorgan's \\
4) & $\lnot A \lor \lnot B \lor C$  & 3 and Associativity of $\lor$
\end{tabular} \\
\end{proof}
\item
* $(A \land \lnot B \land \lnot C) \rightarrow (A \lor \lnot (B \land C))$.
\begin{proof}
\end{proof}
\end{enumerate}


%--
% 2
%--

\item
List the elements of each of the following sets:
\begin{enumerate}
\item
$\mathcal{P} (\{apple, pear, banana\})$
\begin{proof}[Answer:]
$\emptyset, \{apple\}, \{pear\}, \{banana\}, \{apple, pear\}, \{apple, banana\}, \{pear, banana\},\\ \{apple, pear, banana\}$
\end{proof}
\item
* $\mathcal{P} (\{a, b\}) - \mathcal{P} (\{a, c\})$.
\begin{proof}[Answer:]
$\{b\}, \{c\}, \{a, b\}, \{a, c\}$
\end{proof}
\item
$\mathcal{P} (\emptyset)$
\begin{proof}[Answer:]
$\emptyset$
\end{proof}
\item
$\{a, b\} \times \{1, 2, 3\} \times \emptyset$.
\begin{proof}[Answer:]
$\emptyset$
\end{proof}
\item
* $\{x \in \nat: (x \leq 7 \land x \geq 7\}$.
\begin{proof}[Answer:]
$7$
\end{proof}
\item
$\{x \in \nat: \exists y \in \nat \> (y < 10 \land (y + 2 = x))\}$ (where $\nat$ is the set of nonnegative integers).
\begin{proof}[Answer:]
$2, 3, 4, 5, 6, 7, 8, 9, 10, 11$
\end{proof}
\item
$\{x \in \nat: \exists y \in \nat \> (\exists z \in \nat \> ((x = y + z) \land (y < 5) \land z < 4)))\}$.
\begin{proof}[Answer:]
$0, 1, 2, 3, 4, 5, 6, 7, 8, 9$
\end{proof}
\end{enumerate}


%--
% 3
%--

\pagebreak
\item
Prove: $A \cup (B \cap C \cap D) = (A \cup B) \cap (A \cup D) \cap (A \cup C)$.
\begin{proof}
Assume $A \cup (B \cap C \cap D)$. \\ \\
\begin{tabular}{r l l}
1) & $A \cup (B \cap C \cap D)$ & Assumption \\
2) & $A \cup (B \cap (C \cap D))$ & 1 and Association \\
3) & $(A \cup B) \cap (A \cup (C \cap D))$ & 2 and Distribution \\
4) & $(A \cup B) \cap ((A \cup C) \cap (A \cup D))$  & 3 and Distribution \\
5) & $(A \cup B) \cap (A \cup D) \cap (A \cup C)$ & 4 and Association \\
\end{tabular} \\
\end{proof}


%--
% 4
%--

\item
Consider the English sentence, ``If some bakery sells stale bread and some hotel sells flat soda, then the only thing everyone likes is tea.'' This sentence has at least two meanings. Write two (logically different) first-order logic sentences that correspond to meanings that could be assigned to this sentence. Use the following predicates: $P(x)$ is $True$ iff $x$ is a person; $B(x)$ is $True$ iff $x$ is a bakery; $S_B(x)$ is $True$ iff $x$ sells stale bread; $H(x)$ is $True$ iff $x$ is a hotel; $S_S(x)$ is $True$ iff $x$ sells flat soda; $L(x, y)$ is $True$ iff $x$ likes $y$; and $T(x)$ is $True$ iff $x$ is tea.
\begin{proof}[Answer:] Two sentences: \\
1: $(\exists x: B(x) \land S_B(x) )\land (\exists y: H(y) \land S_S(x)) \rightarrow (\forall z \> (\forall p \> P(p) \land L(p, z)) \rightarrow T(z))$ \\
2: $(\exists x: B(x) \land S_B(x) )\land (\exists y: H(y) \land S_S(x)) \rightarrow (\forall p \> P(p) \rightarrow (\forall z \> L(p, z) \rightarrow T(z))) $
\end{proof}


%--
% 5
%--

\item
Let $P$ be the set of positive integers.  Let $L = \{ \mathsf{A, B, \, \ldots, Z} \}$ (i.e., the set of upper case characters in the English alphabet).  Let $T$ be the set of strings of one or more upper case English characters.  Define the following predicates over those sets: \\ \\
\begin{tabular}{c l@{\hspace{2cm}}l}
$\bullet$ & For $x \in L$, & $V(x)$ is $True$ iff $x$ is a vowel.  (The vowels are $\mathsf{A, E, I, O,}$ and $\mathsf{U}$.) \\
$\bullet$ & For $x \in L$ and $n \in P$, & $S(x, n)$ is $True$ iff $x$ can be written in $n$ strokes. \\
$\bullet$ & For $x \in L$ and $s \in T$, & $O(x, s)$ is $True$ iff $x$ occurs in the string $s$. \\
$\bullet$ & For $x, y \in L$, & $B(x, y)$ is $True$ iff $x$ occurs before $y$ in the English alphabet. \\
$\bullet$ & For $x, y \in L$, & $E(x, y)$ is $True$ iff $x = y$.
\end{tabular}
\\

Using these predicates, write each of the following statements as a sentence in first-order logic:
\begin{enumerate}
\item
$\mathsf{A}$ is the only upper case English character that is a vowel and that can be written in three strokes but does not occur in the string $\mathsf{STUPID}$.
\begin{proof}[Answer:]
$\forall x \in L: \> V(x) \land S(x, 3) \land \lnot O(x, \mathsf{STUPID}) \rightarrow E(x, \mathsf{A})$
\end{proof}
\item
There is an upper case English character strictly between $\mathsf{K}$ and $\mathsf{R}$ that can be written in one stroke.
\begin{proof}[Answer:]
$\exists x \in L: B(x, \mathsf{R}) \land \lnot B(x, \mathsf{L}) \land S(x, 1)$
\end{proof}
\end{enumerate}


%--
% 6
%--

\item
* Choose a set $A$ and predicate $P$ and then express the set $\{1, 4, 9, 16, 25, 36, \, \ldots \}$ in the form: \\
$\{x \in A : P(x)\}$.
\begin{proof}[Answer:]
\end{proof}


%--
% 7
%--

\item
Find a set that has a subset but no proper subset.
\begin{proof}[Answer:]
$\emptyset$.  $\emptyset \subseteq \emptyset$ but $\emptyset$ has no proper subset.
\end{proof}


%--
% 9
%--

\pagebreak
\addtocounter{enumi}{1}
\item
Not equal (defined on the integers is (circle all that apply): reflexive, symmetric, transitive. 
\begin{proof}[Answer:]
Symmetric.
\end{proof}

%---
% 11
%---

\addtocounter{enumi}{1}
\item
* Using the definition of $\equiv _p$ (equivalence modulo $p$) that is given in Example A.4, let $R_p$ be a binary relation on $\nat$, defined as follows for any $p \geq 1$: \\
$R_p = \{(a, b): a \equiv _p b \}$ \\
So, for example $R_3$ contains $(0, 0), (0, 3), (6, 9), (1, 4)$, etc., but does not contain $(0, 1), (3, 4)$, etc.
\begin{enumerate}
\item
Is $R_p$ an equivalence relation for every $p \geq 1$? Prove your answer.
\begin{proof}[Answer:]
\end{proof}
\begin{proof}
\end{proof}
\item
If $R_p$ is an equivalence relation how many equivalence classes does $R_p$ induce for a given value of $p$? What are they?  (Any concise description is fine.)
\begin{proof}[Answer:]
\end{proof}
\item
Is $R_p$ a partial order?  A total order?  Prove your answer.
\begin{proof}[Answer:]
\end{proof}
\begin{proof}
\end{proof}
\end{enumerate}

%---
% 13
%---


\addtocounter{enumi}{1}
\item
Are the following sets closed under the following operations?  If not, give an example that proves that they are not and then specify what the closure is.
\begin{enumerate}
\item
The negative integers under subtraction.
\begin{proof}[Answer:]
No.  Ex: $(-3) - (-8) = 5$, which is not in the negative integers.  The integers as a whole are closed under subtraction.
\end{proof}
\item
* The nexative integers under division
\begin{proof}[Answer:]
No.  Ex: ${-10 \over -2} = 5$, which is not a negative integer.  The Real numbers are closed under division.
\end{proof}
\item
The positive integers under exponentiation.
\begin{proof}[Answer:]
Yes.
\end{proof}
\item
The finite sets under Cartesian product.
\begin{proof}[Answer:]
Yes.
\end{proof}
\item
The odd integers under remainder, mod 3.
\begin{proof}[Answer:]
Yes.
\end{proof}
\item
* The rational numbers under addition.
\begin{proof}[Answer:]
Yes.
\end{proof}
\end{enumerate}

%---
% 14
%---

\pagebreak
\item
Give examples to show that:
\begin{enumerate}
\item
The intersection of two countably infinite sets can be finite.
\begin{proof}[Answer:]
Let $A$ be the set of positive integers and $B$ be the set of negative integers.  Both $A$ and $B$ are countably infinite, yet there intersection, $A \cap B$, is $\emptyset$, which is finite.
\end{proof}
\item
The intersection of two countably infinite sets can be countably infinite.
\begin{proof}[Answer:]
$\nat$ is countably infinity, and $\nat \cap \nat = \nat$, which is the same thing.
\end{proof}
\item
The intersection of two uncountable sets can be finite.
\begin{proof}[Answer:]
Let $A$ be the set of positive real numbers and $B$ be the set of negative real numbers.  Both $A$ and $B$ are uncountable, yet there intersection, $A \cap B$, is $\emptyset$, which is finite.
\end{proof}
\item
The intersection of two uncountable sets can be countably infinite.
\begin{proof}[Answer:]
Let $A = [0, 1] \cup [2, 3] \cup [4, 5], \, \ldots$.  Let $B = [1, 2] \cup [3, 4] \cup [5, 6], \, \ldots$.  Both of these sets are uncountable, yet their intersection is the set of positive integers, which is countably infinite. 
\end{proof}
\item
The intersection of two uncountable sets can be uncountable.
\begin{proof}[Answer:]
$\reals$ is countably infinity, and $\reals \cap \reals = \reals$, which is the same thing.
\end{proof}
\end{enumerate}


%---
% 17
%---

\addtocounter{enumi}{2}
\item
Let $\nat$ be the set of nonnegative integers.  Let $A$ be the set of nonnegative integers $x$ such that $x \equiv _3 0$.  Show that $\card{\nat} = \card{A}$.
\begin{proof}
Define a function $f: A \rightarrow \nat$ such that $f(x) = {x \over 3}$.  Since $\forall x, y, z \in \reals ^: y = {x \over 3} \land z = {x \over 3} \rightarrow y = z$, $f$ is one-to-one.  Also, $\forall y \in \nat \> \exists x \in A: f(x) = y$.  We can show this by letting $x = 3y$.  So $f$ is also onto.  Therefore $f$ is a bijection between $\nat$ and $A$, meaning that they have the same cardinality.
\end{proof}


%---
% 21
%---

\addtocounter{enumi}{3}
\item
Use induction to prove each of the following claims:
\begin{enumerate}
\addtocounter{enumii}{1}
\item
$\forall n > 0 \> (n! \geq 2^{n-1})$.  Recall that $0! = 1$ and $\forall n > 0 \> (n! = n(n-1)(n-2) \ldots 1)$.
\begin{proof} Proof by induction: \\ \\
\begin{tabular}{r l}
Base Case: & $(n = 1)$:  $n! = 1 \geq 2^{n-1} = 1$ \\
Induction Hypothesis: & $\exists N \in \ints > 0: N! \geq 2^{N-1}$ \\
\end{tabular}
\\ \\ \\
First, consider that $(N+1)! = (N+1) \times N!$ and $2^{(N+1)-1} = 2 \times 2^{N-1} $.  Now, because $N > 0$, $N+1 \geq 2$, and we also have that $N! \geq 2^{N-1}$ by the induction hypothesis.  Therefore $(N+1)! \geq 2^{(N+1)-1}$.
\end{proof}
\end{enumerate}

%---
% 24
%---

\pagebreak
\addtocounter{enumi}{2}
\item
Prove that the following program computes the function $double(s)$ where, for any string $s$, $double(s) = True$ if $s$ contains at least one pair of adjacent characters that are identical and $False$ otherwise.  Clearly state the loop invariant you are using.
\begin{verbatim}
    double(s:string)=
            found = False.
            for i = 1 to length(s) - 1 do
                if s[i] = s[i+1] then found = True.
            return(found).
\end{verbatim}
\begin{proof}
Proof by loop invariant.\\
\begin{itemize}
\item
$I$ = [The value of $found$ represents the presence of a double in the first $i + 1$ characters of $s$].
\item
On first entry of the loop, $found$ is false and, since we have seen no characters, it reflects the number of doubles found.
\item
Thereafter, if $i = i+1$, $found$ becomes $True$ reflecting the presence of the newly discovered double.  If $i \neq i+1$, $found$ remains false, representing the lack of a double so far.  So $I$ remains true at the end of the loop.
\item
Our loop terminal condition $C$ is that the loop ends when $i = length(s) - 1$.  So combined with $I$, we can see that when the the loop terminates, $found$ represents the presene of a double in the first $length(s) - 1 + 1$ characters of $s$, meaning the whole length of $s$.
\end{itemize}
Therefore this program computes the function $double(s)$.
\end{proof}
\end{enumerate}
\end{document}