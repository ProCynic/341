\documentclass[10pt]{article}

\usepackage{amsfonts, amsthm, amsmath, fullpage, tikz, wrapfig, enumerate}

\newcommand{\card}[1]{\left| #1 \right|}
\newcommand{\nat}{\mathbb{N}}
\newcommand{\ints}{\mathbb{Z}}
\newcommand{\reals}{\mathbb{R}}
\newcommand{\chtitle}[1]{\noindent \vspace{5mm}\textbf{Chapter #1}\vspace{3mm}}
\newcommand{\images}{/home/gparker/classes/341/images}

\begin{document}
\begin{flushleft}
\textbf{\noindent
CS 341 Automata Theory \\
STUDENT NAME - EID \\
Homework 10 \\
Due: Tuesday, March 27}\\
\end{flushleft}
\noindent
This assignment reviews Chapter 13 and covers Chapter 14 and Sections 17.1 - 17.3. \\

\begin{enumerate}[1)]

% ---
% 1
% ---

\item
For each of the following languages $L$, state whether $L$ is regular, context-free but not regular, or not context-free and prove your answer.
\begin{enumerate}[a)]
%a)
\item
\{$w$ : $w = uu^R$ or $w = ua^n$: $n = \card{u}$, $u \in \{a, b\}^*$\}.
\begin{proof}[Answer]
\end{proof}
\begin{proof}[Proof]
\end{proof}

%b)
\item
$\{a^nb^{2n}c^m\} \cap \{a^nb^mc^{2m}\}$.
\begin{proof}[Answer]
\end{proof}
\begin{proof}[Proof]
\end{proof}

%c)
\item
$L^*$, where $L = \{0^*1^i0^*1^i0^* : i \geq 0\}$.
\begin{proof}[Answer]
\end{proof}
\begin{proof}[Proof]
\end{proof}

%d)
\item
$\lnot L_0$, where $L_0 = \{ww : w \in \{a, b\}^*\}$.
\begin{proof}[Answer]
\end{proof}
\begin{proof}[Proof]
\end{proof}

%e)
\item
$\{x \in \{a, b\}^*$ : $\card{x}$ is even and the first half of $x$ has one more a than does the second half\}.
\begin{proof}[Answer]
\end{proof}
\begin{proof}[Proof]
\end{proof}
\end{enumerate}

%---
% 2
%---

\item
Give a decision procedure to answer the following question: given a context-free grammar $G$, does $G$ generate any even length strings?
\begin{proof}[Solution]
\end{proof}

% ---
% 3
% ---

\item
Construct a standard, one-tape Turing machine $M$ to decide the language $L = \{x * y = z$ : $x$, $y$, $z \in 1^+$ and, when  $x$,  $y$, and  $z$ are viewed as unary numbers,  $xy =  z$\}.  For example, the string  $1111*11=11111111 \in L$.  Describe $M$ in the macro language described in Section 17.1.5.
\begin{proof}[Solution]
\end{proof}


% ---
% 4
% ---

\item
Construct a standard 1-tape Turing machine $M$ to compute the function $sub_3$, which is defined as follows:
\begin{tabular}{l c l}
$sub_3(n) =$&$n-3$ &if $n > 2$\\
&$0$&if $n \leq 2$.\\
\end{tabular}\\
Specifically, compute  $sub_3$ of a natural number represented in binary.  For example, on input 10111, $M$ should output 10100.  On input 11101, $M$ should output 11010.  (Hint: you may want to define a subroutine.)
\begin{proof}[Solution]
\end{proof}
\end{enumerate}
\end{document}
