\documentclass[10pt]{article}

\usepackage{amsfonts, amsthm, amsmath, fullpage, tikz, wrapfig, enumerate}

\newcommand{\card}[1]{\left| #1 \right|}
\newcommand{\brackets}[1]{\left< #1 \right>}
\newcommand{\nat}{\mathbb{N}}
\newcommand{\ints}{\mathbb{Z}}
\newcommand{\reals}{\mathbb{R}}
\newcommand{\chtitle}[1]{\noindent \vspace{5mm}\textbf{Chapter #1}\vspace{3mm}}

\begin{document}
\begin{center}
\textbf{
CS 341 Automata Theory \\
Elaine Rich \\
Homework 14 \\
Due: Tuesday, April 23}\\
\end{center}

\noindent
This assignment covers Sections 21.5 - 21.7\\

\noindent
Note: We have skipped Rice's Theorem. So problems 4 - 5 are here just in case you're interested in learning about it. They are optional.

\begin{enumerate}[1)]

% ---
% 1
% ---

\item
For each of the following languages $L$, state whether it is in D, SD/D or not SD. Prove your answer. Do not use Rice's Theorem. If you claim that $L$ is not in SD, first prove that it's not in D (for practice), then prove that it's not in SD. Assume that any input of the form $\brackets{M}$ is a description of a Turing machine.

\begin{enumerate}[a)]

%a
\item
$\{\brackets{M}$ : TM $M$ accepts exactly two strings and they are of different lengths\}.
\begin{proof}[Answer]
\end{proof}
\begin{proof}[Proof]
\end{proof}

%b
\item
$\{\brackets{M, x, y}$ : $M$ accepts \texttt{xy}\}.
\begin{proof}[Answer]
\end{proof}
\begin{proof}[Proof]
\end{proof}

%c
\item
$\{\brackets{M}$ : Turing machine $M$ accepts all even length strings\}.
\begin{proof}[Answer]
\end{proof}
\begin{proof}[Proof]
\end{proof}

%d
\item
$\{\brackets{M}$ : $M$ rejects exactly three strings that start with \texttt{a}\}
\begin{proof}[Answer]
\end{proof}
\begin{proof}[Proof]
\end{proof}

%e
\item
$\{\brackets{M_a, M_b}$ : $L(M_a) - L(M_b) = \emptyset$\}.
\begin{proof}[Answer]
\end{proof}
\begin{proof}[Proof]
\end{proof}
\end{enumerate}


%---
% 2
%---

\item
Prove that $\mathrm{TM_{REG}}$ is not in SD.
\begin{proof}[Proof]
\end{proof}

% ---
% 3
% ---

\item
For any nonempty alphabet $\Sigma$, let $L$ be any decidable language other than $\emptyset$ or $\Sigma ^*$. Prove that $L \leq _M \lnot L$.
\begin{proof}[Proof]
\end{proof}


% ---
% 4
% ---

\item
* Do the other half of the proof of Rice's Theorem, i.e., show that the theorem holds if $P(\emptyset) = True$. (Hint: use a reduction that is not a mapping reduction.)
\begin{proof}[Proof]
\end{proof}

% ---
% 5
% ---

\item
* Use Rice's Theorem to prove that \{$\brackets{M}$ : Turing machine $M$ accepts at least two odd length strings\} is not in D.
\end{enumerate}
\begin{proof}[Proof]
\end{proof}
\end{document}
