\documentclass[10pt]{article}

\usepackage{amsfonts, amsthm, amsmath, fullpage, tikz, wrapfig, enumerate}

\newcommand{\card}[1]{\left| #1 \right|}
\newcommand{\brackets}[1]{\left< #1 \right>}
\newcommand{\nat}{\mathbb{N}}
\newcommand{\ints}{\mathbb{Z}}
\newcommand{\reals}{\mathbb{R}}
\newcommand{\chtitle}[1]{\noindent \vspace{5mm}\textbf{Chapter #1}\vspace{3mm}}

\begin{document}
\begin{center}
\textbf{
CS 341 Automata Theory \\
Elaine Rich \\
Homework 14 \\
Due: Tuesday, April 23}\\
\end{center}

\noindent
This assignment covers Sections 21.5 - 21.7\\

\noindent
Note: We have skipped Rice's Theorem. So problems 4 - 5 are here just in case you're interested in learning about it. They are optional.

\begin{enumerate}[1)]

% ---
% 1
% ---

\item
For each of the following languages $L$, state whether it is in D, SD/D or not SD. Prove your answer. Do not use Rice's Theorem. If you claim that $L$ is not in SD, first prove that it's not in D (for practice), then prove that it's not in SD. Assume that any input of the form $\brackets{M}$ is a description of a Turing machine.
\begin{enumerate}[a)]

%a
\item
$\{\brackets{M}$ : TM $M$ accepts exactly two strings and they are of different lengths\}.
\begin{proof}[Answer]
Not $\in$ SD.
\end{proof}
\begin{proof}[Proof]$ $\\
To prove that this language is not $\in$ D:\\
Create a reduction $R$ from $\brackets{M, w}$ to $\brackets{M}$ as follows.
\begin{enumerate}[1.]
\item
Create a description $\brackets{M_\#}$ of a machine $M_\#$ that does:
\begin{enumerate}
\item[1.1]
Erase the tape.

\item[1.2]
Write $w$ on the tape.

\item[1.3]
Run $M$ on $w$.

\item[1.4]
If $x$ is \texttt{a} or \texttt{aa} accept.
\end{enumerate}
\item
$R$ returns $M_\#$\\
\end{enumerate}

Assume there is a machine $Oracle(\brackets{M})$ that decides this language.  Then for any language-string pair $\brackets{M, w}$ consider $C = oracle(R(\brackets{M, w}))$.\\

If $\brackets{M, w} \in$ H: $M_\#$ accepts exactly two strings. Oracle accepts.
If $\brackets{M, w} \not \in$ H: $M_\#$ accepts no strings.  Oracle rejects.

Therefore $C$ decides H so $Oracle$ must not exist.

\newpage
To prove that this language is not $\in$ SD:\\

Create a reduction $R$ from $\brackets{M, w}$ to $\brackets{M}$ as follows.
\begin{enumerate}[1.]
\item
Create a description $\brackets{M_\#}$ of a machine $M_\#$ that does:
\begin{enumerate}
\item[1.1]
If $x$ is \texttt{a} or \texttt{aa} accept.

\item[1.2]
Erase the tape.

\item[1.3]
Write $w$ on the tape.

\item[1.4]
Run $M$ on $w$.

\item[1.5]
Accept.
\end{enumerate}
\item
$R$ returns $M_\#$\\
\end{enumerate}

Assume there is a machine $Oracle(\brackets{M})$ that decides this language.  Then for any language-string pair $\brackets{M, w}$ consider $C = oracle(R(\brackets{M, w}))$.\\

If $\brackets{M, w} \in \lnot$ H: $M_\#$ Accepts exactly two strings and loops on all others.  Oracle accepts.\\
If $\brackets{M, w} \not \in \lnot$ H: $M_\#$ Accepts all strings.  Oracle rejects.\\

Therefore $C$ semi-decides $\lnot$H so $Oracle$ must not exist.
\end{proof}

%b
\item
$\{\brackets{M, x, y}$ : $M$ accepts $xy$\}.
\begin{proof}[Answer]
In SD but not in D.
\end{proof}
\begin{proof}[Proof]$ $\\
Let $R$ be a reduction from $\brackets{M, w}$ to $\brackets{M, x, y}$ as follows:
\begin{enumerate}[1.]
\item
Create a description $\brackets{M_\#}$ of a machine $M_\#$ that does:
\begin{enumerate}
\item[1.1]
Erase the tape.

\item[1.2]
Write $w$ on the tape.

\item[1.3]
Run $M$ on $w$.

\item[1.4]
Accept.
\end{enumerate}
\item
Return $\brackets{M_\#, \epsilon, \epsilon}$
\end{enumerate}


Assume by way of contradiction that there exists some machine $Oracle(\brackets{M, x, y})$ that decides whether $M$ accepts $xy$.  Then for any machine description string pair $\brackets{M, w}$ let $C = oracle(R(\brackets{M, w}))$. Now there are two cases:\\

If $\brackets{M, w} \in$ H: $M_\#$ accepts all strings, so it accepts $xy$.  Oracle accepts.\\
If $\brackets{M, w} \not \in$ H: $M_\#$ accepts no strings, so it does not accept $xy$.  Oracle rejects.\\

So $C$ decides H.  Therefore $Oracle$ does not exist.
\end{proof}

\pagebreak
%c
\item
$\{\brackets{M}$ : Turing machine $M$ accepts all even length strings\}.
\begin{proof}[Answer]
Not $\in$ SD.
\end{proof}
\begin{proof}[Proof]$ $\\
This is the proof that this language is not in D:\\

Let $R$ be a reduction from $\brackets{M, w}$ to $\brackets{M}$ as follows:
\begin{enumerate}[1.]
\item
Create a description $\brackets{M_\#}$ of a machine $M_\#$ that does:
\begin{enumerate}
\item[1.1]
Erase the tape.

\item[1.2]
Write $w$ on the tape.

\item[1.3]
Run $M$ on $w$.

\item[1.4]
Accept.
\end{enumerate}
\item
Return $\brackets{M_\#}$
\end{enumerate}

Assume by way of contradiction that there exists some machine $Oracle(\brackets{M})$ that decides whether $M$ accepts all even length strings.  Then for any machine description string pair $\brackets{M, w}$ let $C = oracle(R(\brackets{M, w}))$. Now there are two cases:\\

If $\brackets{M, w} \in$ H: $M_\#$ accepts all strings, so it accepts all even length strings.  Oracle accepts.\\
If $\brackets{M, w} \not \in$ H: $M_\#$ accepts no strings, so it does not accept all even length strings.  Oracle rejects.\\

So $C$ decides H.  Therefore $Oracle$ does not exist.\\

\vspace{1cm}
\noindent
Now the proof that this language is not in SD:\\
Let $R$ be a reduction from $\brackets{M, w}$ to $\brackets{M}$ as follows:
\begin{enumerate}[1.]
\item
Create a description $\brackets{M_\#}$ of a machine $M_\#(x)$ that does:
\begin{enumerate}
\item[1.1]
Copy the input $x$ onto a second tape.

\item[1.2]
Erase the tape.

\item[1.3]
Write $w$ on the tape.

\item[1.4]
Run $M$ on $w$ for $\card{x}$ steps.

\item[1.5]
If $M$ did not halt naturally, accept.
\end{enumerate}
\item
Return $\brackets{M_\#}$.
\end{enumerate}

Assume by way of contradiction that there exists some machine $Oracle(\brackets{M})$ that decides whether $M$ accepts all even length strings.  Then for any machine description string pair $\brackets{M, w}$ let $C = oracle(R(\brackets{M, w}))$. Now there are two cases:\\

If $\brackets{M, w} \in \lnot$ H: $M_\#$ accepts all strings, so it accepts all even length strings.  Oracle accepts.\\
If $\brackets{M, w} \not \in \lnot$ H: $M_\#$ does not accept on even length strings of length less than $\card{x}$.  Oracle rejects.\\

So $C$ decides $\lnot$ H.  Therefore $Oracle$ does not exist.
\end{proof}

\pagebreak
%d
\item
$\{\brackets{M}$ : $M$ rejects exactly three strings that start with \texttt{a}\}
\begin{proof}[Answer]
Not $\in$ SD.
\end{proof}
\begin{proof}[Proof]$ $\\
This is the proof that this language is not in D:\\

Let $R$ be a reduction from $\brackets{M, w}$ to $\brackets{M}$ as follows:
\begin{enumerate}[1.]
\item
Create a description $\brackets{M_\#}$ of a machine $M_\#$ that does:
\begin{enumerate}
\item[1.1]
Copy $x$ to a second tape.

\item[1.2]
Erase the tape.

\item[1.3]
Write $w$ on the tape.

\item[1.4]
Run $M$ on $w$.

\item[1.5]
If $x$ is \texttt{a} or \texttt{aa} or\texttt{aaa}, reject.

\item[1.6]
Accept.
\end{enumerate}
\item
Return $\brackets{M_\#}$
\end{enumerate}

Assume by way of contradiction that there exists some machine $Oracle(\brackets{M})$ that decides this language.  Then for any machine description string pair $\brackets{M, w}$ let $C = oracle(R(\brackets{M, w}))$. Now there are two cases:\\

If $\brackets{M, w} \in$ H: $M_\#$ rejects exactly three strings that start with \texttt{a}  Oracle accepts.\\
If $\brackets{M, w} \not \in$ H: $M_\#$ rejects no strings.  Oracle rejects.\\

So $C$ decides H.  Therefore $Oracle$ does not exist.\\

\vspace{1cm}
\noindent
Now the proof that this language is not in SD:\\
Let $R$ be a reduction from $\brackets{M, w}$ to $\brackets{M}$ as follows:
\begin{enumerate}[1.]
\item
Create a description $\brackets{M_\#}$ of a machine $M_\#(x)$ that does:
\begin{enumerate}
\item[1.1]
If $x$ is \texttt{a} or \texttt{aa} or\texttt{aaa}, reject.

\item[1.2]
Erase the tape.

\item[1.3]
Write $w$ on the tape.

\item[1.4]
Run $M$ on $w$.

\item[1.5]
Reject.
\end{enumerate}
\item
Return $\brackets{M_\#}$.
\end{enumerate}

Assume by way of contradiction that there exists some machine $Oracle(\brackets{M})$ that semi-decides the language.  Then for any machine description string pair $\brackets{M, w}$ let $C = oracle(R(\brackets{M, w}))$. Now there are two cases:\\

If $\brackets{M, w} \in \lnot$ H: $M_\#$ rejects exactly three strings that start with \texttt{a}.  Oracle accepts.\\
If $\brackets{M, w} \not \in \lnot$ H: $M_\#$ rejects all strings.  Oracle rejects.\\

So $C$ decides $\lnot$ H.  Therefore $Oracle$ does not exist.
\end{proof}

\pagebreak
%e
\item
$\{\brackets{M_a, M_b}$ : $L(M_a) - L(M_b) = \emptyset$\}.
\begin{proof}[Answer]
Not $\in$ SD.
\end{proof}
\begin{proof}[Proof]$ $\\
This is the proof that this language is not in D:\\

Let $R$ be a reduction from $\brackets{M, w}$ to $\brackets{M}$ as follows:
\begin{enumerate}[1.]
\item
Create descriptions $\brackets{M_1}$ and $\brackets{M_1}$ of machines $M_1$ and $M_2$ where $M_1$ always accepts and $M_2$ does:\\

\begin{enumerate}
\item[1.1]
Erase the tape.

\item[1.3]
Write $w$ on the tape.

\item[1.4]
Run $M$ on $w$.

\item[1.5]
Accept.
\end{enumerate}
\item
Return $\brackets{M_1, M_2}$
\end{enumerate}

Assume by way of contradiction that there exists some machine $Oracle(\brackets{M})$ that decides this language.  Then for any machine description string pair $\brackets{M, w}$ let $C = oracle(R(\brackets{M, w}))$. Now there are two cases:\\

If $\brackets{M, w} \in$ H: $M_2$ accepts everyhing and $L(M_1) - L(M_2) = \emptyset$.  Oracle accepts.
If $\brackets{M, w} \not \in$ H: $M_2$ accepts nothing and $L(M_1) - L(M_2) = \Sigma ^*$.  Oracle rejects.

So $C$ decides H.  Therefore $Oracle$ does not exist.\\

\vspace{1cm}
\noindent
Now the proof that this language is not in SD:\\
Let $R$ be a reduction from $\brackets{M, w}$ to $\brackets{M}$ as follows:
\begin{enumerate}[1.]
\item
Create descriptions $\brackets{M_1}$ and $\brackets{M_1}$ of machines $M_1$ and $M_2$ where $M_2$ always rejects and $M_1$ does:\\
\begin{enumerate}
\item[1.1]
Erase the tape.

\item[1.2]
Write $w$ on the tape.

\item[1.3]
Run $M$ on $w$.

\item[1.4]
Accept.
\end{enumerate}
\item
Return $\brackets{M_1, M_2}$
\end{enumerate}

Assume by way of contradiction that there exists some machine $Oracle(\brackets{M})$ that semi-decides the language.  Then for any machine description string pair $\brackets{M, w}$ let $C = oracle(R(\brackets{M, w}))$. Now there are two cases:\\

If $\brackets{M, w} \in \lnot$ H: $M_1$ accepts nothing and $L(M_1) - L(M_2) = \emptyset$.  Oracle accepts.\\
If $\brackets{M, w} \not \in \lnot$ H: $M_2$ accepts everything and $L(M_1) - L(M_2) = \Sigma ^*$.  Oracle rejects.\\

So $C$ decides $\lnot$ H.  Therefore $Oracle$ does not exist.
\end{proof}
\end{enumerate}


\pagebreak
%---
% 2
%---

\item
Prove that $\mathrm{TM_{REG}}$ is not in SD.
\begin{proof}[Proof]$ $\\
Let $R$ be a reduction from $\brackets{M, w}$ to $\brackets{M}$ as follows:
\begin{enumerate}[1.]
\item
Create a description $\brackets{M_\#}$ of a machine $M_\#(x)$ that does:
\begin{enumerate}
\item[1.1]
Write $x$ to the second tape.

\item[1.2]
Erase the tape.

\item[1.3]
Write $w$ on the tape.

\item[1.4]
Run $M$ on $w$.

\item[1.5]
If $x \in a^nb^n$ accept.
\end{enumerate}
\item
Return $\brackets{M_\#}$.
\end{enumerate}

Assume by way of contradiction that there exists some machine $Oracle(\brackets{M})$ that semi-decides the language.  Then for any machine description string pair $\brackets{M, w}$ let $C = oracle(R(\brackets{M, w}))$. Now there are two cases:\\

If $\brackets{M, w} \in \lnot$ H: $L(M_\#) = \emptyset$ which is regular. Oracle accepts.\\
If $\brackets{M, w} \not \in \lnot$ H: $L(M_\#) = a^nb^n$ which is regular. Oracle rejects.\\

So $C$ decides $\lnot$ H.  Therefore $Oracle$ does not exist.
\end{proof}

% ---
% 3
% ---

\item
For any nonempty alphabet $\Sigma$, let $L$ be any decidable language other than $\emptyset$ or $\Sigma ^*$. Prove that $L \leq _M \lnot L$.
\begin{proof}[Proof]
\end{proof}


% ---
% 4
% ---

\item
* Do the other half of the proof of Rice's Theorem, i.e., show that the theorem holds if $P(\emptyset) = True$. (Hint: use a reduction that is not a mapping reduction.)
\begin{proof}[Proof]
\end{proof}

% ---
% 5
% ---

\item
* Use Rice's Theorem to prove that \{$\brackets{M}$ : Turing machine $M$ accepts at least two odd length strings\} is not in D.
\begin{proof}[Proof]
\end{proof}
\end{enumerate}
\end{document}
