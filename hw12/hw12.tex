\documentclass[10pt]{article}

\usepackage{amsfonts, amsthm, amsmath, fullpage, tikz, wrapfig, enumerate}

\newcommand{\card}[1]{\left| #1 \right|}
\newcommand{\nat}{\mathbb{N}}
\newcommand{\ints}{\mathbb{Z}}
\newcommand{\reals}{\mathbb{R}}
\newcommand{\chtitle}[1]{\noindent \vspace{5mm}\textbf{Chapter #1}\vspace{3mm}}
\newcommand{\images}{/home/gparker/classes/341/images}

\begin{document}
\begin{flushleft}
\textbf{\noindent
CS 341 Automata Theory \\
STUDENT NAME - EID\\
Homework 12 \\
Due: Tuesday, April 10}\\
\end{flushleft}
\noindent
This assignment covers Chapter 20.\\

\begin{enumerate}[1)]

% ---
% 1
% ---

\item
* Let $L_1, L_2, \ldots , L_k$ be a collection of languages over some alphabet $\Sigma$ such that:
\begin{itemize}
\item
For all $i \neq j$, $L_i \cap L_j = \emptyset$.
\item
$L_1 \cup L_2 \cup \dots \cup L_k = \Sigma ^*$.
\item
$\forall i$ ($L_i$ is in SD).
\end{itemize}
Prove that each of the languages $L_1$ through $L_k$ is in $D$.
\begin{proof}[Proof]
\end{proof}

%---
% 2
%---

\item
If $L_1$ and $L_3$ are in D and $L_1 \subseteq L_2 \subseteq L_3$, what can we say about whether $L_2$ is in D?
\begin{proof}[Answer]
\end{proof}

% ---
% 3
% ---

\item
Let $M$ be a Turing machine that lexicographically enumerates the language $L$. Prove that there exists a Turing
machine $M'$ that decides $L^R$.
\begin{proof}[Proof]
\end{proof}


% ---
% 4
% ---

\item
Construct a standard one-tape Turing machine $M$ to enumerate the language $A^nB^n$. Assume that $M$ starts with
its tape equal to $\square$. Also assume the existence of the printing subroutine $P$, defined in Section 20.5.1.
\begin{proof}[Solution]
\end{proof}


% ---
% 5
% ---

\item
Recall the function $mix$, defined in Example 8.23. Neither the regular languages nor the context-free languages
are closed under $mix$. Are the decidable languages closed under $mix$? Prove your answer.
\begin{proof}[Answer]
\end{proof}
\begin{proof}[Proof]
\end{proof}


% ---
% 6
% ---

\item
Let $\Sigma = \{a, b\}$. Consider the set of all languages over $\Sigma$ that contain only even length strings.
\begin{enumerate}[a)]
%a
\item
How many such languages are there?
\begin{proof}[Answer]
\end{proof}


%b
\item
How many of them are semidecidable?
\begin{proof}[Answer]
\end{proof}
\end{enumerate}
\end{enumerate}
\end{document}
