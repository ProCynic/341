\documentclass[10pt]{article}

\usepackage{amsfonts, amsthm, amsmath, fullpage, tikz, wrapfig, enumerate}

\newcommand{\card}[1]{\left| #1 \right|}
\newcommand{\nat}{\mathbb{N}}
\newcommand{\ints}{\mathbb{Z}}
\newcommand{\reals}{\mathbb{R}}
\newcommand{\chtitle}[1]{\noindent \vspace{5mm}\textbf{Chapter #1}\vspace{3mm}}
\newcommand{\images}{/home/gparker/classes/341/images}

\begin{document}
\begin{center}
\textbf{
CS 341 Automata Theory \\
Elaine Rich \\
Homework 7 \\
Due: Tuesday, February 28}\\
\end{center}
\noindent
This assignment covers Chapter 11. \\

\begin{enumerate}[1)]
% ---
% 1
% ---

\item
Let $\Sigma = \{a, b\}$. For the languages that are defined by each of the following grammars, do each of the following:
\begin{center}
\begin{enumerate}[i.]
%i. 
\item
List five strings that are in $L$. 

%ii.
\item
List five strings that are not in $L$.

%iii.
\item
Describe $L$ concisely.  You can use regular expressions, expressions using variables (e.g., $a^nb^n$, or set theoretic expressions (e.g., $\{x: \ldots\}$)

%iv.
\item
Indicate whether or not $L$ is regular.  Prove your answer.
\end{enumerate}
\end{center}
\begin{enumerate}[a)]
%a)
\item
$S \rightarrow \texttt{a}S \mid S\texttt{b} \mid \epsilon$

%b)
\item
$S \rightarrow \texttt{a}S\texttt{a} \mid \texttt{b}S\texttt{b} \mid a \mid b$

%c)
\item
* $S \rightarrow \texttt{a}S \mid \texttt{b}S \mid \epsilon$

%d)
\item
* $S \rightarrow \texttt{a}S \mid \texttt{a}S\texttt{b}S \mid \epsilon$
\end{enumerate}

%---
% 2
%---
a) 
b) * 00101101
\item
Consider the following grammar $G : S \rightarrow \texttt{0}S\texttt{1} \mid SS \mid \texttt{10}$. Show a parse tree produced by $G$ for each of the following strings:
\begin{enumerate}[a)]
%a
\item
\texttt{010110}

%b
\item
* \texttt{00101101}
\end{enumerate}

% ---
% 3
% ---

\item
Let $G$ be the grammar of Example 11.12.  Show a third parse tree that $G$ can produce for the string $(())()$.

% ---
% 4
% ---

\item
Consider the following context free grammar G:
\begin{align*}
S &\rightarrow \texttt{a}S\texttt{a}\\
S &\rightarrow T\\
S &\rightarrow \epsilon\\
T &\rightarrow \texttt{b}T\\
T &\rightarrow \texttt{c}T\\
T &\rightarrow \epsilon\\
\end{align*}
One of these rules is redundant and could be removed without altering $L(G)$.  Which one?


% ---
% 5
% ---

\item
Using the simple English grammar that we showed in Example 11.6, show two parse trees for each of the following sentences.  In each case, indicate which parse tree almost certainly corresponds to the intended meaning of the sentence:
\begin{enumerate}[a)]
%a)
\item
\texttt{The bear shot Fluffy with the rifle}.

%b)
\item
\texttt{Fluffy likes the girl with the chocolate}.
\end{enumerate}


% --
% 6
% --
\item
Show a context-free grammar for each of the following languages $L$:
\begin{enumerate}[a)]
%a
\item
$\{a^ib^j\ :\ 2i = 3j + 1\}$.

%b
\item
* $\{w \in \{a, b\}^*\ :\ \#_a(w) = 2 \#_b(w)\}$.

%c
\item
$\{w \in \{a, b\}^*\ :\ w = w^R\}$.

%d
\item
$\{w \in \{a, b\}^*$ : every prefix of $w$ has at least as many $a$’s as $b$’s\}.

%e
\item
$\{a^mb^nc^pd^q\ :\ m, n, p, q \geq 0$ and $m + n = p + q\}$.

%f
\item
$\{b_i\#b_{i+1}^R\ :\ b_i$ is the binary representation of some integer  $i,\ i \geq 0$, without leading zeros\}. (For example 101\#011 $\in L$.)
\end{enumerate}

% --
% 7
% --
\item
* Let $G$ be the ambiguous expression grammar of Example 11.14.  Show at least three different parse trees that can be generated from $G$ for the string \texttt{id+id*id*id}.

% --
% 8
% --
\item
Consider the unambiguous expression grammar $G$ of Example 11.19.
\begin{enumerate}[a)]
%a)
\item
Trace the derivation of the string \texttt{id+id*id*id} in $G$.
%b)
\item
Add exponentiation $(**)$ and unary minus $(-)$ to  $G$, assigning the highest precedence to unary minus, followed by exponentiation, multiplication, and addition, in that order.
\end{enumerate}

% --
% 9
% --
\item
Let $L = \{w \in \{a, b, \cup, \epsilon, (, ), *, +\}^*$ : $w$ is a syntactically legal regular expression\}.  
\begin{enumerate}[a)]
%a)
\item
Write an  unambiguous context-free grammar that generates  L.  Your grammar should have a structure 
similar to the arithmetic expression grammar G that we presented in Example 11.19.  It should create parse 
trees that:
\begin{itemize}
\item
Associate left given operators of equal precedence, and
\item
Correspond to assigning the following precedence levels to the operators (from highest to lowest):
\begin{enumerate}[1)]
\item
$*$ and $+$

\item
concatenation

\item
$\cup$
\end{enumerate}
\end{itemize}
%b)
\item
Show the parse tree that your grammar will produce for the string $(a \cup b)ba^*$.
\end{enumerate}

% --
% 10
% --
\item
* In Appendix I.3.1, we present a simplified grammar for URIs (Uniform Resource Identifiers), the names that we use to refer to objects on the Web.  
\begin{enumerate}[a)]
%a)
\item
Using that grammar, show a parse tree for:\\
\texttt{https://www.mystuff.wow/widgets/fradgit\#sword}
%b)
\item
Write a regular expression that is equivalent to the grammar that we present.
\end{enumerate}

% --
% 11
% --
\item
Consider the grammar $G = (\{S, A, B, T, a, c\}, \{a, c\}, R, S)$, where $R = \{S \rightarrow AB,\ S \rightarrow BA,\ A \rightarrow aA,\ A \rightarrow ac,\ B \rightarrow Tc,\ T \rightarrow aT,\ T \rightarrow a\}$.
\begin{enumerate}[a)]
%a)
\item
Show that $G$ is ambiguous.  
%b)
\item
Find an equivalent grammar that is not ambiguous.
\end{enumerate}
\end{enumerate}
\end{document}
