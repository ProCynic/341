\documentclass[10pt]{article}

\usepackage{amsfonts, amsthm, fullpage}

\newcommand{\card}[1]{\left| #1 \right|}
\newcommand{\nat}{\mathbb{N}}
\newcommand{\ints}{\mathbb{Z}}
\newcommand{\reals}{\mathbb{R}}
\newcommand{\chtitle}[1]{\noindent \vspace{5mm}\textbf{Chapter #1}\vspace{3mm}}   


\begin{document}
\begin{flushleft}
\textbf{Geoffrey Parker - grp352 \\
CS 341 Automata Theory \\
Homework 2 \\
Due Thursday, Jan. 26}
\end{flushleft}
This assignment covers Chapters 2 - 4. \\

\chtitle{2}

\begin{enumerate}

% ---
% 2.1
% ---

\item
Consider the language $L = \{1^n2^n: n > 0\}$.  Is the string $122$ in $L$?
\begin{proof}[Answer:]
No.
\end{proof}

% ---
% 2.2
% ---

\item
Let $L_1$ = $\{a^nb^n: n > 0\}$.  Let $L_2 = \{c^n: n > 0\}$.  For each of the following strings, state whether or not it is an element of $L_1L_2$:
\begin{enumerate}

%a
\item
$\epsilon$.
\begin{proof}[Answer:]
No.
\end{proof}

%b
\item
$aabbcc$.
\begin{proof}[Answer:]
Yes.
\end{proof}

%c
\item
$abbcc$.
\begin{proof}[Answer:]
No.
\end{proof}

%d
\item
$aabbcccc$.
\begin{proof}[Answer:]
Yes.
\end{proof}
\end{enumerate}

% ---
% 2.3
% ---

\item
Let $L_1 = \{peach, apple, cherry\}$ and $L_2 = \{pie, cobbler, \epsilon\}$.  List the elements of $L_1L_2$ in lexicographic order.
\begin{proof}[Answer:]
$apple,\ cherry,\ peach,\ applecobbler,\ applepie,\ cherrycobbler,\ cherrypie,\ peachcobbler,\ peachpie$.
\end{proof}

% ---
% 2.4
% ---

\item
* Let $L = \{w \in \{a, b\}^* : \card{w} \equiv _3 0\}$.  List the first six elements in a lexicographic enumeration of $L$.
\begin{proof}[Answer:]
$\epsilon,\ aaa,\ aab,\ aba,\ abb,\ baa$ 
\end{proof}


\pagebreak
% ---
% 2.5
% ---

\item
Consider the language $L$ of all strings drawn from the alphabet $\{a, b\}$ with at least two different substrings of length 2.
\begin{enumerate}

%a
\item
Describe $L$ by writing a sentence of the form $L = \{w \in \Sigma ^* : P(w)\}$, where $\Sigma$ is a set of symbols and $P$ is a first-order logic formula.  You may use the function $\card{s}$ to return the length of $s$.  You may use all the standard relational symbols (e.g., $=$,  $\neq$, $<$, etc.), plus the predicate  $Substr(s,  t)$, which is $True$ iff $s$ is a substring of $t$.
\begin{proof}[Answer:]
$L = \{w \in \{a, b\}^* : \exists a\ \exists b (Substr(a, w) \land Substr(b, w) \land \card{a} = 2 \land \card{b} = 2 \land a \neq b)\}$.
\end{proof}

%b
\item
List the first six elements of a lexicographic enumeration of $L$.
\begin{proof}[Answer:]
$aab,\ aba,\ abb,\ baa,\ bab,\ bba$.
\end{proof}
\end{enumerate}

% ---
% 2.6
% ---

\item
* For each of the following languages $L$, give a simple English description of $L$.  Show two strings that are in $L$ and two that are not (unless there are fewer than two strings in $L$ or two not in $L$, in which case show as many as possible).
\begin{enumerate}

%a
\item
$L = \{w \in \{a, b\}^*$ : exactly one prefix of $w$ ends in $a\}$. 
\begin{proof}[Answer:]
\end{proof}

%b
\item
$L = \{w \in \{a, b\}^*$ : all prefixes of $w$ end in $a\}$.
\begin{proof}[Answer:]
\end{proof}

%c
\item
$L = \{w \in \{a, b\}^*$ : $\exists x \in \{a, b\}^+ (w = axa)\}$.
\begin{proof}[Answer:]
\end{proof}
\end{enumerate}

% ---
% 2.7
% ---

\item
Are the following sets closed under the following operations?  If not, what are their respective closures?
\begin{enumerate}

%a
\item
The language $\{a, b\}$ under concatenation.
\begin{proof}[Answer:]
Yes.
\end{proof}

%b
\item
The odd length strings over the alphabet $\{a, b\}$ under Kleene star.
\begin{proof}[Answer:]
No.  The closure is $\{a, b\}^*$
\end{proof}

%c
\item
$L = \{w \in \{a, b\}^*\}$ under reverse.
\begin{proof}[Answer:]
Yes.
\end{proof}

%d
\item
$L = \{w \in \{a, b\}^*$ : $w$ starts with $a\}$ under reverse.
\begin{proof}[Answer:]
No.  The closure is $\{w \in \{a, b\}^*$ : $w$ starts and ends with $a\}$
\end{proof}

%e
\item
$L = \{w \in \{a, b\}^*$ : $w$ ends in $a\}$ under concatenation.
\begin{proof}[Answer:]
Yes.
\end{proof}
\end{enumerate}
\end{enumerate}
 
\pagebreak
\chtitle{3}
\begin{enumerate}
\addtocounter{enumi}{1}

% ---
% 3.2
% ---

\item
Consider the optical character recognition (OCR) problem: Given an array of black and white pixels, and a set of characters, determine which character best matches the pixel array. Formulate this problem as a language recognition problem.
\begin{proof}[Answer:]
$L = \{<a, c>$ : the character $c$ is the best match for the pixel array $a\}$.
\end{proof}

% ---
% 3.3
% ---

\item
Consider the language $A^nB^nC^n = \{a^nb^nc^n: n \geq 0\}$, discussed in Section 3.3.3.  We might consider the following design for a PDA to accept $A^nB^nC^n$:  As each $a$ is read, push two $a$'s onto the stack.  Then pop one $a$ for each $b$ and one $a$ for each $c$.  If the input and the stack both become empty, accept.  Otherwise reject.  Why doesn’t this 
work?
\begin{proof}[Answer:]
This PDA doesn't work because it would accept the string $aabcc$, which is not part of the language $A^nB^nC^n$.
\end{proof}

% ---
% 3.4
% ---

\item
* Define a PDA-2 to be a PDA with two stacks (instead of one).  Assume that the stacks can be manipulated independently and that the machine accepts iff it is in an accepting state and both stacks are empty when it runs out of input.  Describe the operation of a PDA-2 that accepts $A^nB^nC^n = \{a^nb^nc^n : n \geq 0\}$.  (Note: we will see, in Section 17.5.2, that the PDA-2 is equivalent to the Turing machine in the sense that any language that can be 
accepted by one can be accepted by the other.)
\begin{proof}[Answer:]
\end{proof}
\end{enumerate}


\chtitle{4}
\begin{enumerate}

% ---
% 4.1
% ---

\item
Given a Java program $p$ and the input $0$, consider the question,``Does $p$ ever output anything?''
\begin{enumerate}

%a
\item
Describe a semidecision procedure that answers this question.
\begin{proof}[Answer:]
Compile $p$ and run it with the input $0$.  If there is output return yes, if it exits without output, return no.
\end{proof}

%b
\item
Is there an obvious way to turn your answer to part (a) into a decision procedure?
\begin{proof}[Answer:]
No, this is equivalent to the halting problem.
\end{proof}
\end{enumerate}

% ---
% 4.2
% ---

\item
Let $L = \{w \in \{a, b\}^*: w = w^R\}$.  What is $chop(L)$? 
\begin{proof}[Answer:]
$chop(L) = \{w \in L: w\ is\ even\}$.
\end{proof}

% ---
% 4.3
% ---

\item
Are the following sets closed under the following operations?  Prove your answer.  If a set is not closed under the operation, what is its closure under the operation?

\begin{enumerate}

%a
\item
* The odd length strings over the alphabet $\{a, b\}$ under concatenation.
\begin{proof}[Answer:]
No.  The closure is $\{a, b\}^+$
\end{proof}
\begin{proof}
$a$ and $b$, both strings in the language, concatenate to produce $ab$, which is not in the language.
\end{proof}

%b
\item
$L = \{w \in \{a, b\}^* : w$ ends in $a\}$ under the function odds, defined on strings as follows: \linebreak $odds(s)$ = the string that is formed by concatenating together all of the odd numbered \linebreak characters of $s$.  (Start numbering the characters at 1.)  For example, $odds(ababbbb) = aabb$.
\begin{proof}[Answer:]
No. The closure is $\{w \in \{a, b\}^* : w$ ends in $a$ and $\card{w}$ is odd\}.
\end{proof}
\begin{proof}
The string $ba$ is in L, but $odds(ba) = b$ is not.
\end{proof}

%c
\item
FIN (the set of finite languages) under the function $oddsL$, defined on languages as follows: $oddsL(L) = \{w : \exists x \in L (w = odds(x))\}$ 
\begin{proof}[Answer:]
Yes.
\end{proof}
\begin{proof}
The subset of a finite language must also be finite, and $oddsL(L) \subseteq L$, so FIN is closed under $oddsL$.
\end{proof}

%d
\item
* INF (the set of infinite languages) under the function $oddsL$.
\begin{proof}[Answer:]
\end{proof}
\begin{proof}
\end{proof}

%e
\item
FIN under the function $maxstring$, defined in Example 8.22.
\begin{proof}[Answer:]
Yes.
\end{proof}
\begin{proof}
The subset of a finite language must also be finite, and $maxstring(L) \subseteq L$, so FIN is closed under $maxstring$.
\end{proof}

%f
\item
INF under the function $maxstring$. 
\begin{proof}[Answer:]
No.  The closure is INF $\cup$ FIN
\end{proof}
\begin{proof}
The language $a^*b^* \in$ INF.  $maxstring(a^*b^*) = \emptyset$, which is finite, and so not in INF.
\end{proof}
\end{enumerate}

% ---
% 4.4
% ---

\item
Describe a program, using choose, to do one of the following, or select another interesting puzzle and do this problem with it:
\begin{enumerate}

%a
\item
Play Sudoku.

%b
\item
Solve Rubik's Cube \textregistered .
\end{enumerate}
\begin{proof}[Answer:]
\begin{verbatim}


sudoku(starting-board):
    /* Assume subset(board1, board2) returns True if board2 has all the same numbers
    as board1 in all the same places and false otherwise.  board2 may have additional
    numbers corresponding to the blank spaces in board1. */
    
    /* Assume complete-boards is the set of all completely filled legal sudoku boards. */
    
    return choose(x from complete-boards: subset(starting-board, x)).
\end{verbatim}
\end{proof}
\end{enumerate}
\end{document}
