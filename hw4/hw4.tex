\documentclass[10pt]{article}

\usepackage{amsfonts, amsthm, fullpage, tikz, wrapfig, enumerate}

\newcommand{\card}[1]{\left| #1 \right|}
\newcommand{\nat}{\mathbb{N}}
\newcommand{\ints}{\mathbb{Z}}
\newcommand{\reals}{\mathbb{R}}
\newcommand{\chtitle}[1]{\noindent \vspace{5mm}\textbf{Chapter #1}\vspace{3mm}}
\newcommand{\images}{/home/gparker/classes/341/images}

\begin{document}
\begin{center}
\textbf{
CS 341 Automata Theory \\
Elaine Rich \\
Homework 4 \\
Due Tuesday, February 7}\\
\end{center}
\noindent
This assignment covers Sections 5.9 - 5.11 and Chapter 6 \\

\noindent
\textbf{Sections 5.9 - 5.11}
\begin{enumerate}[1)]
\addtocounter{enumi}{14}
% ---
% 15
% ---

\item
Consider the problem of counting the number of words in a text file that may contain letters plus any  of the following characters:
\[<blank>\ <linefeed>\ <end-of-file>\ ,\ .\ ;\ :\ ?\ !\]
Define a word to be a string of letters that is preceded by either the beginning of the file or some non-letter character and that is followed by some non-letter character.  For example, there are 11 words in the following text:
\begin{flushleft}

\parindent 2cm

\texttt{\noindent
\hspace{2cm}The $<blank>$ $<blank>$ cat $<blank>$ $<linefeed>$\\
saw $<blank>$ the $<blank>$ $<blank>$ $<blank>$ rat $<linefeed>$\\
$<blank>$ with\\
$<linefeed>$ a $<blank>$ hat $<linefeed>$\\
on $<blank>$ the $<blank>$ $<blank>$ mat $<end-of-file>$
}
\end{flushleft}
Describe a very simple finite-state transducer that reads the characters in the file one at a time and solves the wordcounting problem.  Assume that there exists an output symbol with the property that, every time it is generated, an 
external counter gets incremented.

\addtocounter{enumi}{1}
% ---
% 17
% ---

\item
\includegraphics[scale=.1]{\images /hw4barcode.png}
Real bar code systems are more complex than the one we sketched in the book.  They 
must be able to encode all ten digits, for example.  There are several industry-standard 
formats for bar codes, including the common UPC code found on nearly everything we 
buy.  Search the web.  Find the encoding scheme used for UPC codes.  Describe a finite 
state transducer that reads the bars and outputs the corresponding decimal number.  You do 
not need to write out every state.  Show some at the beginning.  Then describe, in English, 
the structure of the rest of the machine.



% ---
% 18
% ---

\item
Extend the description of the Soundex FSM that was started in Example 5.33 so that it can assign a code to the 
name Pfifer.  Remember that you must take into account the fact that every Soundex code is made up of exactly four 
characters.

% ---
% 19
% ---

\item
* Consider the weather/passport HMM of Example 5.37.  Trace the execution of the Viterbi and forward algorithms to answer the following questions:

\begin{enumerate}[a)]
%a
\item
Suppose that the report $\#\#\#L$ is received from Athens.  What was the most likely weather during the time of the report?
 
%b
\item
Is it more likely that $\#\#\#L$ came from London or from Athens?
\end{enumerate}


\addtocounter{enumi}{2}
% --
% 22
% --

\item
In I.1.2, we describe the Alternating Bit Protocol for handling message transmission in a network.  Use the FSM that describes the sender to answer the question, ``Is there any upper bound on the number of times a message may be retransmitted?''
\end{enumerate}

\textbf{Chapter 6}

\begin{enumerate}[1)]
% ---
% 1
% ---

\item
Describe in English, as briefly as possible, the language defined by each of these regular expressions:
\begin{enumerate}[a)]
%a
\item
$(b \cup ba) (b \cup a)^* (ab \cup b)$.
\end{enumerate}

% ---
% 2
% ---

\item
Write a regular expression to describe each of the following languages:
\begin{enumerate}[a)]
\addtocounter{enumii}{1}
%b
\item
$\{w \in \{a, b\}^*$ : $w$ does not end in $ba\}$.

\addtocounter{enumii}{1}
%d
\item
$\{w \in \{0, 1\}^*$ : $w$ corresponds to the binary encoding, without leading $0$'s, of natural numbers that are evenly divisible by $4\}$.

%e
\item
$\{w \in \{0, 1\}^*$ : $w$ corresponds to the binary encoding, without leading $0$'s, of natural numbers that are powers of $4\}$.

\addtocounter{enumii}{2}
%h
\item
$\{w \in \{0, 1\}^*$ : $w$ does not have \texttt{001} as a substring\}.

\addtocounter{enumii}{7}
%p
\item
* $\{w \in \{0, 1\}^*$ : $w$ does not have \texttt{001} as a substring\}.

%q
\item
* $\{w \in \{a, b\}^*$ : $w$ contains no more than two occurrences of the substring \texttt{aa}\}.
\end{enumerate}

% ---
% 3
% ---
\item
Simplify each of the following regular expressions:
\begin{enumerate}[a)]
%a
\item
$(a \cup b)^* (a \cup \epsilon) b^*$.

%b
\item
$(\emptyset ^* \cup b) b^*$.

%c
\item
* $(a \cup b)^*a^* \cup b$.

%d
\item
* $((a \cup b)^*)^*$.

%e
\item
$((a \cup b)^+)^*$.

%f
\item
$a ( (a \cup b)(b \cup a) )^* \cup a ( (a \cup b) a )^* \cup a ( (b \cup a) b )^*$.
\end{enumerate}

% ---
% 4
% ---
\item
For each of the following expressions $E$, answer the following three questions and prove your answer:
\begin{center}
\begin{enumerate}[(i)]
%i
\item
Is $E$ a regular expression?

%ii
\item
If $E$ is a regular expression, give a simpler regular expression.

%iii
\item
Does $E$ describe a regular language?
\end{enumerate}
\end{center}

\begin{enumerate}[a)]
\addtocounter{enumii}{1}
%b
\item
$(a^+a^nb^n)$.

\addtocounter{enumii}{1}
%d
\item
$(((ab) \cup c)^* \cap (b \cup c^*))$.
\end{enumerate}

\addtocounter{enumi}{4}
% ---
% 9
% ---

\item
* Consider the following FSM $M$:
\begin{center}
\includegraphics[scale=.45]{\images /hw4fsm9.png}
\end{center}

\begin{enumerate}[a)]
%a
\item
Show a regular expression for $L(M)$.

%b
\item
Describe $L(M)$ in English.
\end{enumerate}

\addtocounter{enumi}{5}
% ---
% 15
% ---
\item
Consider the following FSM M:

\begin{center}
%\includegraphics[scale=.45]{\images /hw4fsm15.png}
\end{center}

\begin{enumerate}[a)]
%a
\item
Show a regular expression for $L(M)$.

%b
\item
Show a DFSM that accepts $\lnot L(M)$.
\end{enumerate}

\addtocounter{enumi}{2}
% ---
% 18
% ---

\item
Let $\Sigma = \{a, b\}$.  Let $L = \{\epsilon, a, b\}$. Let $R$ be a relation defined on $\Sigma ^*$ as follows: $\forall xy\ (xRy$ iff $y = xb)$.  Let $R'$ be the reflexive, transitive closure of $R$.  Let $L' = \{x : \exists y \in L (yR'x)\}$.  Write a regular expression for $L'$.

\addtocounter{enumi}{1}
% ---
% 20
% ---
\item
For each of the following statements, state whether it is $True$ or $False$.  Prove your answer.

\begin{enumerate}[a)]
\addtocounter{enumii}{6}
%g
\item
If $\alpha$ and $\beta$ are any two regular expressions, then $(\alpha \cup \beta)^* = \alpha(\beta\alpha \cup \alpha)$.

%h
\item
If $\alpha$ and $\beta$ are any two regular expressions, then $(\alpha\beta)^*\alpha = \alpha(\beta\alpha)^*$.
\end{enumerate}
\end{enumerate}
\end{document}
