\documentclass[10pt]{article}

\usepackage{amsfonts, amsthm, amsmath, fullpage, tikz, wrapfig, enumerate}

\newcommand{\card}[1]{\left| #1 \right|}
\newcommand{\brackets}[1]{\left< #1 \right>}
\newcommand{\nat}{\mathbb{N}}
\newcommand{\ints}{\mathbb{Z}}
\newcommand{\reals}{\mathbb{R}}
\newcommand{\chtitle}[1]{\noindent \vspace{5mm}\textbf{Chapter #1}\vspace{3mm}}

\begin{document}
\begin{flushleft}
\textbf{\noindent
CS 341 Automata Theory \\
STUDENT NAME - EID\\
Homework 13 \\
Due: Tuesday, April 17}\\
\end{flushleft}

\noindent
This assignment covers Sections 21.1 - 21.3\\

\begin{enumerate}[1)]

% ---
% 1
% ---

\item
In Appendix E.3, we describe a straightforward use of reduction that solves a grid coloring problem by reducing it to a graph problem.  Given the grid $G$ shown here:
\begin{center}
\includegraphics[scale=.15]{images/p1.png}
\end{center}
\begin{enumerate}[a)]

%a
\item
Show the graph that corresponds to $G$.
\begin{proof}[Solution]
\begin{align*}
G' &= \{V, E\}\\
V &= \{A, B, C, D, 1, 2, 3, 4\}\\
E &= \{(A, 3), (A, 4), (B, 1), (B, 3), (B, 4), (C, 4), (D, 2)\}
\end{align*}
\end{proof}

%b
\item
Use the graph algorithm we describe to find a coloring of $G$.
\begin{proof}[Solution]
Start with $B$ and color each edge alternately.
\begin{align*}
Red &= \{(B, 1), (B, 4)\}\\
Blue &= \{(B, 3)\}
\end{align*}
Now do vertex $3$:
\begin{align*}
Red &= \{(B, 1), (B, 4), (A, 3)\}\\
Blue &= \{(B, 3)\}
\end{align*}
Now vertex $A$:
\begin{align*}
Red &= \{(B, 1), (B, 4), (A, 3)\}\\
Blue &= \{(B, 3), (A, 4)\}
\end{align*}
Now vertex $4$: (it has one of each already, so pick arbitrarily.)
\begin{align*}
Red &= \{(B, 1), (B, 4), (A, 3)\}\\
Blue &= \{(B, 3), (A, 4), (C, 4)\}
\end{align*}
And now the last edge:
\begin{align*}
Red &= \{(B, 1), (B, 4), (A, 3), (D, 2)\}\\
Blue &= \{(B, 3), (A, 4), (C, 4)\}
\end{align*}
\end{proof}
\end{enumerate}


%---
% 2
%---

\item
In this problem, we consider the relationship between $H$ and a very simple language $\{a\}$.
\begin{enumerate}[a)]

%a
\item
Show that $\{\texttt{a}\}$ is \emph{mapping} reducible to $H$.  
\begin{proof}[Solution]
Construct a turing machine $M$ which halts and accepts if the square to the right of the read head is an \texttt{a} and otherwise loops.  The problem of recognizing $\{\texttt{a}\}$ is now equivalent to the problem of deciding whether $M$ halts.
\end{proof}

%b
\item
Is it possible to reduce $H$ to $\{\texttt{a}\}$?  Prove your answer.
\begin{proof}[Answer]
No.
\end{proof}
\begin{proof}[Proof]
We know that $\{\texttt{a}\}$ is a regular language, and is thus decidable.  So if there were a way to reduce $H$ to $\{\texttt{a}\}$, then $H$ would be decidable.  However we know that $H$ is not decidable, so such a reduction must not exist.
\end{proof}
\end{enumerate}

% ---
% 3
% ---

\item
Show that $H_{ALL}$ is not in $D$ by reduction from $H$.
\begin{proof}[Solution]
Assume by way of contradiction that a Turing Machine $Oracle$ that decided $H_{ALL}$ existed.  Now define a reduction $R(\brackets{M, w})$ which constructs a description of a  new machine $M\#$ that does the following:
\begin{enumerate}[1.]
%1.
\item
Erases its input tape.

%2.
\item
Writes $w$ onto the tape.

%3.
\item
Passes control to $M$.
\end{enumerate}

Now when $M(w)$ halts $Oracle(R(\brackets{M, w}))$ accepts and when $M(w)$ does not halt $Oracle(R(\brackets{M, w}))$ rejects.
\end{proof}


% ---
% 4
% ---

\item
For each of the following languages $L$, state whether or not it is in $D$.  Prove your answer.  Assume that any input of the form $\brackets{M}$ is a description of a Turing machine.
\begin{enumerate}[a)]

%a
\item
$\{\brackets{M}\ :\ \texttt{ab} \in L(M)\}$.
\begin{proof}[Answer]
No.
\end{proof}
\begin{proof}[Proof]
Define $R(\brackets{M, w})$, a mapping reduction from $H$ to this language as follows:
\begin{enumerate}[1.]
\item
Construct the description $\brackets{M\#}$ of a new Turing machine $M\#$ that, on input $x$, operates as follows:
\begin{enumerate}
\item[1.1]
Erase the tape.

\item[1.2]
Write $w$ on the tape.

\item[1.3]
Run $M$ on $w$.

\item[1.4]
Accept.
\end{enumerate}

\item
Return $\brackets{M\#, w}$.
\end{enumerate}
If there were a machine to decide this language, call it $Oracle$ then $C = Oracle(R(\brackets{M, w}))$ would decide $H$.\\
If $\brackets{M, w} \in H$:  $M$ halts on $w$, so $M\#$ accepts everything.  In particular, it accepts \texttt{ab}.  $Oracle(\brackets{M\#})$ accepts.\\
If $\brackets{M, w} \not \in H$:  $M$ halts on $w$, so $M\#$ does not halt on any input.  In particular, it does not halt on \texttt{ab}.  $Oracle(\brackets{M\#})$ rejects.
\end{proof}

%b
\item
$\{\brackets{M, w}$ : TM $M$, on input $w$, begins by moving right one square onto $w$.  Then it never moves off $w$\}.
\begin{proof}[Answer]
No.
\end{proof}
\begin{proof}[Proof]
\end{proof}

%c
\item
$\{\brackets{M}$ : there exists a string $w$ such that $\card{w} < \card{\brackets{M}}$ and that $M$ accepts $w$\}.
\begin{proof}[Answer]
No.
\end{proof}
\begin{proof}[Proof]
Let $R$ be a mapping reduction from $H$ to this language as follows:\\
$R(\brackets{M, w}) = $\\
\begin{enumerate}[1.]
\item
Construct the description $\brackets\{M\#\}$ of a new Turine machine $M\#(x)$ that, on input $x$, operates as follows:
\begin{enumerate}
\item[1.1]
Erase the tape.
\item[1.2]
Write $w$ to the tape.
\item[1.3]
Run $M$ on $w$
\item[1.4]
Accept.
\end{enumerate}
\item
Return $\brackets{M\#}$.
\end{enumerate}
Assume by way of contradiction that there exists a machine to decide this language, call it $Oracle$.
If $\brackets{M, w} \in H$: then $M\#$ will accept on all input, including $\epsilon$.  So there exists at least one string with length less than $\brackets{M}$ on which $M\#$ accepts.  $Oracle$ accepts.\\
If $\brackets{M, w} \not \in H$: then $M\#$ will not halt, and thus not accept, on all input.  So there does not exists at least one string with length less than $\brackets{M}$ on which $M\#$ accepts.  $Oracle$ rejects.\\
Therefore $Oracle(R(\brackets{M, w}))$ decides $H$.  However we know that no such machine exists, so $Oracle$ must not exist.
\end{proof}
\end{enumerate}

% ---
% 5
% ---

\item
In Appendix J.2, we proved Theorem J.1, which tells us that the safety of even a very simple security model is undecidable, by reduction from $H_\epsilon$.  Show an alternative proof that reduces $A = \{\brackets{M, w}$ :  $M$ is a Turing machine and $w \in L(M)$\} to the language Safety.
\begin{proof}[Proof]
\end{proof}
\end{enumerate}
\end{document}
